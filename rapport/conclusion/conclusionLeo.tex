Lors de ce projet, j'ai appris de nombreuses choses. Tout d'abord, j'ai appris à coder en C tout en étant rigoureux et en s'efforçant de faire un code lisible. En effet, comme nous étions plusieurs à travailler sur ce projet il était nécessaire d'être lisible dans mon code pour que mes camarades puissent facilement relire et comprendre ce que j'ai codé. De plus, ce projet fut l'occasion pour consolider toutes mes connaissances apprises en cours, notamment sur l'utilisation de l'allocation dynamique, des pointeurs, de la manipulation de SDD et de fichier. Ensuite j'ai découvert différents outils de travail tel que git. Étant la première fois que j'utilise git j'ai appris à l'utiliser et à comprendre comment cet outil fonctionne. J'ai donc compris toute la force de cet outil qui est nécessaire pour travailler en groupe sur un projet informatique. Puis, j'ai aussi découvert l'outil valgrind, qui est un logiciel très efficace pour corriger les fuites de mémoires et diverses autres erreurs comme les erreurs de segmentations.\\\\
Durant ce projet, le travail de groupe était très important. En effet, ce projet étant conséquent, il était indispensable de se partager les tâches. J'ai donc appris à suivre les directives d'un chef de projet et à produire mon travail en temps et en heure pour que mes camarades puissent continuer leurs codes. Par exemple, il fallait que je code les tests unitaires du TAD correcteurs orthographiques pour que Ruth puisse l'implémenter en pouvant tester l'efficacité de son code.\\\\
Ce projet fût donc une grande expérience qui à consolider mes capacités en programmation et en travail de groupe.