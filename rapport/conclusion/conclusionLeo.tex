Lors de ce projet, j'ai appris de nombreuses choses. Tout d'abord, j'ai appris à coder en C tout en étant rigoureux et en m'efforçant de faire un code lisible. En effet, comme nous étions plusieurs à travailler sur ce projet il était nécessaire d'être lisible dans mon code pour que mes camarades puissent facilement relire et comprendre ce que j'ai codé. De plus, ce projet fut l'occasion pour moi de consolider toutes mes connaissances apprises en cours, notamment sur l'utilisation de l'allocation dynamique, des pointeurs, de la manipulation de SDD et de fichiers. Ensuite, j'ai découvert différents outils de travail tel que GIT. Sachant que c'était la première fois que j'utilisais cet outil, j'ai appris à l'utiliser et à comprendre comment celui-ci fonctionnait. J'ai pu donc découvrir tous les atouts de GIT qui sont nécessaires au travail en groupe sur un projet informatique. Puis, j'ai aussi découvert l'outil Valgrind, qui est un logiciel très efficace pour corriger les fuites de mémoires et diverses autres erreurs comme les erreurs de segmentation.\\\\
De plus, durant ce projet le travail de groupe était très important. En effet, clui-ci étant conséquent, il était indispensable de se partager les tâches. J'ai donc appris à suivre les directives d'un chef de projet et à produire mon travail en temps et en heures pour que mes camarades puissent continuer leurs codes. Par exemple, il fallait que je code les tests unitaires du TAD CorrecteurOrthographique pour que Ruth puisse l'implémenter en pouvant tester l'efficacité de son code.\\\\
Ce projet fût donc une grande expérience qui a consolidé mes capacités en programmation ainsi qu'en travail de groupe.
