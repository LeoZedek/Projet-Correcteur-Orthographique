Ce projet de corrceteur orthographique m'a permis de découvrir la conception et le dévellopement d'un logiciel informatique par le biai du language C.*
J'avais déjà étudier somairement le C en premier année post BAC mais je mettais arrété à des programmes n'ayant qu'un fichier (ou alors plusieurs fichier mais avec un dévellopement guidé et un makefile donné)
J'ai aussi apris l'utilisation de l'environnement git, l'utilisation poussé du language \Latex, la conception de documentation grace à doxygen. 
Il m'a aussi permis de découvrir la gestion d'équipe et un peu de pédagogie pour expliquer le fonctionnement de certain outils à  mes collègues.\\
Nous avons eu du mal à nous lancer au début du projet, l'apprentissage de l'environnement git, de la gestion du dépot et la mise en place du rapport, 
pour m'a part ont pris du temps sur l'analyse.\\
