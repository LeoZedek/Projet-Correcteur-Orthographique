Ce projet de correcteur orthographique m'a permis de découvrir la conception et le développement d'un logiciel informatique par le biais du langage C.
J'avais déjà étudié sommairement le C en première année post BAC mais je m'étais arrêté à des programmes n'ayant qu'un fichier 
(ou alors plusieurs fichiers mais avec un développement guidé et un makefile donné).
J'ai aussi apris l'utilisation de l'environnement git, l'utilisation poussé du language \LaTeX, la conception de documentation grace à doxygen.
D'autres outils m'ont aussi aidé lors du débuggage comme Valgrind ou l'option de compilation -fsainitize=address pour détecter les fuites de mémoires. 
Ce projet m'a aussi permis de découvrir la gestion d'équipe ainsi que de la pédagogie pour expliquer le fonctionnement de certain outils à  mes collègues. 
J'ai aussi confirmé des compétences apprises plus tôt en TP comme la réalisation d'un makefile, l'utilisation du framework CUnit.\\

Néanmoins, nous avons eu du mal à nous lancer au début du projet, l'apprentissage de l'environnement git, de la gestion du dépôt et la mise en place du rapport, 
pour ma part ont pris du temps sur l'analyse.\\

Une fois le développement lancé, le plus dur pour moi a été de coordonner tout le monde pour que les tests soient prêts avant que la personne implémente sa partie attribuée.
Dans mon cas je me suis retrouvé à développer sans tests et il a été très dur de repérer les erreurs avant de les avoir.\\

La deuxième difficulté a été l'optimisation du programme : une fois ma partie terminée et les tests unitaires validés, j'ai lancé le programme en ajoutant les 300 000 mots. N'ayant pas vu l'arrêt du programme, j'ai dû l'arréter avant en mettant un affichage : j'ajoutais 100 000 mots en 5 minutes, ce qui n'était pas acceptable.
